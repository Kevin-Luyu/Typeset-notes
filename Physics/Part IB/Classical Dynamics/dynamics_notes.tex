\documentclass{article}
\usepackage[thmmarks, framed]{ntheorem} % ntheorem must come before amsmath, or the cross-reference will not be working
\usepackage[utf8]{inputenc}
\usepackage{graphicx} 
\usepackage{empheq} % box aligned equations
\usepackage{booktabs}
\usepackage[a4paper, portrait, margin=1in]{geometry}
\usepackage{amsmath}
\usepackage{amsfonts}
\usepackage{mathtools}
\usepackage{bm} % bold greek letters 
\usepackage{physics}
\usepackage{xcolor}
% needed for framed theorems
\usepackage{framed} % or, "mdframed"
%
\usepackage{bm}
\newcommand*\widefbox[1]{\fbox{\hspace{2em}#1\hspace{2em}}}
\title{NST IB Physics B Classical Dynamics Notes}
\date{Lent, 2023}
\author{Yu Lu}
\begin{document}
\maketitle
\section{Rigid body dynamics}
\subsection{Dynamical quantities and the inertial tensor}
The general motion of a rigid body in three dimensions can be quite complicated. However, it is always possible to define an instantaneous angular momentum vector (Euler's theorem) about which the body momentarily rotates. Therefore, we could always decompose the motion of a rigid body into translation of its centre of mass and the rotation in the CM frame (only in this frame can we neglect the angular momentum of the translational motion). In the centre of mass frame, the body can rotate, precess, or nutate around axes through the origin. 

In the CM frame, external forces acting uniformly across the body can be neglected, since that only brings about translational motion of the CM via $M \ddot{\mathbf{R} } = \mathbf{F_0}. $ The total angular momentum 
\[
    \mathbf{J} = \sum\limits_{\text{body}} m \mathbf{r} \times \mathbf{v} 
    = \sum\limits_{\text{body}} m \mathbf{r} \times (\boldsymbol{\mathbf{\omega}}  \times \mathbf{r} )
\]
can be related to the instantaneous angular velocity via a tensor relationship
\[\boxed{
    \mathbf{J} = \overline{\overline{I}} \cdot \boldsymbol{\mathbf{\omega}},}
\]
where the \textbf{moment of inertia tensor} is determined as 
\[
    \overline{\overline{I}} = 
    \begin{pmatrix}
        \Sigma m(y^{2} + z^{2} ) & -\Sigma m x y &  - \Sigma m x z \\
        - \Sigma  m x y & \Sigma m(x^{2} +z^{2} ) &  -\Sigma myz \\
        -\Sigma mxz & -\Sigma myz &  \Sigma m(x^{2} +y^{2} ). \\
    \end{pmatrix}
\]
From its definition $T = \Sigma m\mathbf{v} ^{2}, $ the kinetic energy can be determined as a quadratic form:
\[\boxed{
    T = \frac{1}{2} \mathbf{J} \cdot \boldsymbol{\mathbf{\omega}} 
    = \frac{1}{2} \boldsymbol{\mathbf{\omega}}^{\top} \cdot \overline{\overline{I}} \cdot \boldsymbol{\mathbf{\omega}}.}
\]

Since the inertial tensor $\overline{\overline{I}} $ is by construction real and symmetric, it has real eigenvalues and mutually orthogonal eigenvectors, making it always possible to be diagonalised. The mutually orthogonal eigenvectors are called \textbf{principal axes} of the rigid body and are fixed in the body frame. In the principal axes coordinates, the inertial tensor is diagonal:
\[
    \overline{\overline{I}} = 
    \begin{pmatrix}
        I_1 & 0 &  0 \\
        0 & I_2 &  0 \\
        0 & 0 &  I_3, \\
    \end{pmatrix}
\]
and expressions for the angular momentum and the kinetic energy are simple:
\[
    T = \frac{1}{2} (I_1 \omega_1^2 + I_2 \omega_2^2 + I_3 \omega_3^2), 
    \qquad 
    \mathbf{J} = I_1 \omega_1 \hat{\mathbf{e_1} } + I_2 \omega_2 \hat{\mathbf{e_2} } + I_3 \omega_3 \hat{\mathbf{e_3} }.  
\]
In the special case that $I_1 = I_2$, the body is called a symmetric top and only $\hat{\mathbf{e_3} } $ is uniquely determined. The other two axes can be any two orthogonal axes in the plane perpendicular to the extraordinary axis $\hat{\mathbf{e_3} }.$

It can be seen from direct calculation that $I_i + I_j \geq I_k,$ where equality is taken iff the body is a laminar in the $ij$ space (perpendicular axes theorem). Also, for an axis parallel to to one of the principal axis through the origin, the parallel axes theorem tells that
\[
    \boxed{
        I = I_{\text{CM}} + M r^2,
    }
\]
where $r$ is the distance from the CM to the axis of interest. This, once again, stresses the unique standing of CM in a rigid body. 

In $\boldsymbol{\mathbf{\omega}}$-space, the surface of constant kinetic energy is an ellipsoid (\textbf{inertia ellipsoid}), with $\mathbf{J} $ being the surface normal and $\boldsymbol{\mathbf{\omega}}$ being the position vector. The three axes of the ellipsoid coincide with the principal axes of the body (though the length is proportional to $1/\sqrt{I_i}$), so the inertia ellipsoid rotates with the body in space. 

It may be useful to memorise that the MoI of a sphere is $I = \frac{2}{5} M a^{2} $ and that of a thin rod through its CoM is $I = \frac{1}{12} M l^2. $

\subsection{Equation of motion}
It is vitally important to recognise that the body frame, in which most calculation involving the inertial tensor is done, is a rotating frame with non-uniform angular velocity. Thus, we need to establish the connection of rate of change of vectors in the body frame to the lab frame (the space frame). The rate of change of a basis vector (or any vector fixed in the body frame) is 
\begin{equation}
    \frac{\mathrm{d}}{\mathrm{d}t} \hat{\mathbf{e_i} } = \boldsymbol{\mathbf{\omega}}  \times \hat{\mathbf{e_i} }. \label{eq:rate-of-change}
\end{equation}
This can be regarded as the definition of $\boldsymbol{\mathbf{\omega}}$ and can be rationalised by noting the fact that $\frac{\mathrm{d}}{\mathrm{d}t} (\left\lVert \hat{\mathbf{e_i} } \right\rVert^{2}  ) = \hat{\dot{\mathbf{e_i}}}\cdot \hat{\mathbf{e_i} } = 0$ from normalisation or exploiting the unitarity of rotation (see Goldstein for a formal discussion). Then, from product rule of differentiation, the rate of change in the inertial frame $S_0$ can be connected to that in the body frame $S$ as 
\[
    \boxed{
        \left[ \frac{\mathrm{d}}{\mathrm{d}t} \right]_{S_0}
        = \left[ \frac{\mathrm{d}}{\mathrm{d}t} \right]_{S} + \boldsymbol{\mathbf{\omega}}  \times .
    }
\]

The rotational Newton's 2nd law 
\[
    \boxed{ 
        \mathbf{G} = \frac{\mathrm{d}\mathbf{J} }{\mathrm{d}t}
    }
\]
holds in the space frame, and can be converted to \textbf{Euler's equations} in the body frame as 
\begin{empheq}[box = \widefbox]{align*}
    G_1 &= I_1 \dot{\omega}_1 + (I_3 - I_2) \omega_2 \omega_3, \\ 
    G_2 &= I_2 \dot{\omega}_2 + (I_1 - I_3) \omega_3 \omega_1, \\ 
    G_3 &= I_3 \dot{\omega}_3 + (I_2 - I_1) \omega_1 \omega_2.  
\end{empheq}
The same equation could also be derived by considering how the rotation of principal axes changes the angular velocity.

Due to the curvature of the inertial ellipsoid, the 1-axis and the 3-axis are local extrema of angular momentum while the intermediate axis is a saddle point. Thus, a small perturbation in $\mathbf{L}$ which was previously aligned with the 2-axis gives rise to unstable wobbling of the body, assuming the kinetic energy is conserved. The stability could be more formally shown by Euler's equations. 

In other cases where the magnitude of angular momentum is conserved while the kinetic energy is dissipated, aligning $L$ with the major axis minimises the total energy and is therefore most stable ($T = \frac{J^2}{2I}$). 

\subsection{The symmetric top}
A symmetric top is defined as a rigid body with $I_1 = I_2 \neq I_3,$ where the equality usually comes from antisymmetry around the 3-axis (also called the extraordinary axis). Consequently, the angular momentum $\mathbf{J} $ and the angular velocity $\boldsymbol{\mathbf{\omega}} $ are collinear iff there is no component of $\mathbf{L}$ along the 3-direction. Furthermore, only the 3-axis is unique and we can chose any set of orthogonal axes in the plane normal to the extraordinary axis. This often greatly simplifies the problem. 

\subsubsection{Free precession}
In the absence of external torques (such as an unsupported heavy body), the angular momentum of the symmetric top is conserved. The free motion of a symmetric top can be characterised by its rotation along the $\boldsymbol{\mathbf{\omega}}$ vector and the precession of $\boldsymbol{\mathbf{\omega}}$ around the fixed $\mathbf{L} $. This is a consequence of $\mathbf{G} = \textbf{0} $ and $I_2 - I_1 = 0$, leading to $\omega_3 = 0$. Together with the conservation of kinetic energy, the only permitted trace of $\boldsymbol{\mathbf{\omega}}$ is a cone (known as the space cone, see Poinsot's construction)

In the body frame of the top, the 3-axis is fixed while $\mathbf{J}$ and $\boldsymbol{\mathbf{\omega}}$ precesses around it with the same body frequency $\Omega_b.$ The motion in the body frame is governed by Euler's equations:
\[
    \begin{dcases}
        I_1 \dot{\omega_1} &= (I_1 - I_3) \omega_2 \omega_3   \\
        I_1 \dot{\omega_2} &= (I_3 - I_1) \omega_1 \omega_3
    \end{dcases}
    \implies 
    \omega_1 + i \omega_2 = A e^{i \Omega_b},
\]
where $\boxed{\Omega_b = \frac{I_1 - I_3}{I_1}\omega_3}. $

In the space frame, the angular momentum $\mathbf{L}$ is fixed while the $\boldsymbol{\mathbf{\omega}} $ and $\hat{\mathbf{e_3}}$ precesses with a precession frequency $\Omega_s.$ From the definition of $\boldsymbol{\mathbf{\omega}} $ and $\mathbf{L},$ we could note the relationship 
\[
    \boldsymbol{\mathbf{\omega}} = \frac{\mathbf{L}}{I_1} - \Omega_b \hat{\mathbf{e_3}},  
\]
indicating that the three vectors are always coplanar. Then, since the body axes are fixed in the body frame, we could write its rate of change in the space frame as 
\[
    \frac{\mathrm{d}\hat{\mathbf{e_3} }}{\mathrm{d}t} = 
    \boldsymbol{\mathbf{\omega}} \times \hat{\mathbf{e_3} }
    = \frac{\mathbf{L} }{I_1} \times \hat{\mathbf{e_3} }.
\]
Comparing with the rate of change in different frames, we could interpret this as precession of $\hat{\mathbf{e_3} }$ around $\mathbf{L}$ with a precession rate $\boxed{\Omega_s = \frac{L}{I_1}}.$

Therefore, in the space frame, the angular velocity $\boldsymbol{\mathbf{\omega}} $ of a free symmetric top precesses around its angular momentum $\mathbf{L} $ where the body axis $\hat{\mathbf{e_3} }$ precesses around $\boldsymbol{\mathbf{\omega}}.$ The overall motion can be regarded as rolling of the body cone around the space cone. Note also that the precession rate determined by an initial observer is the space frequency. The rolling without slipping condition could also relate the two frequency:
\[
    \Omega_b \sin{\theta_b} = \Omega_s \sin{\theta_s},
\]
where $\theta_b$ is the angle between $\boldsymbol{\mathbf{\omega}}$ and $\hat{\mathbf{e_3} }$ while $\theta_s$ is the angle between $\boldsymbol{\mathbf{\omega}}$ and $\mathbf{L}.$ 

\subsubsection{Gyroscope precession and nutation}
A more general treatment of rigid body involves a convenient set of generalised coordinates $\{\theta, \phi, \chi\}$ called the Euler angles, where $\chi$ describes the rotation about the 3-axis, $\phi$ describes the precession of the angular momentum around the precession axis, and $\chi$ describes the nutation in the angle between the 3-axis and the precession axis. Apparently, such description would be much harder in the case of a asymmetric top where precession could happen around any axis. For a symmetric top, however, we have straightforwardly 
\[
    \boldsymbol{\mathbf{\omega}} = \dot{\phi} \hat{\mathbf{e_z}} + \dot{\theta} \hat{\mathbf{e_z}} + \dot{\chi } \hat{\mathbf{e_3}}, 
\]
and the conversion to an orthogonal system is easy. 

For a gyroscope, the torque due to gravity $G = mg \hat{\mathbf{e_r}} \times \mathbf{z}$ has no component in $\hat{\mathbf{e_z}} $ and $\hat{\mathbf{e_3}} $ directions, giving two first integrals:
\[
    J_3 = I_3 (\dot{\chi } + \dot{\phi } \cos{\theta }), \quad 
    J_z = J_3 \cos{\theta } + J_2 \sin \theta. 
\]
Simplifying the two equations gives, rather unsurprisingly, 
\[
    \boxed{
    \dot{\phi } = \Omega_s, \quad 
    \dot{\chi } = \Omega_b, }
\]
which simply confirms our perception of the free precession as the body cone rolling on the space cone. Conservation of energy then gives $E = I_1 \dot{\theta }^{2} /2 + U_{\mathrm{eff}}(\theta),$ where 
\[
    U_{\mathrm{eff}} = \frac{(J_z - J_3 \cos \theta )^{2} }{2I_1 \sin^{2} \theta } + mgh \cos \theta + \frac{J_3 ^{2} }{2I_3}. 
\]
Solving for $U_{\theta } = 0$ gives the range of nutation angles. To achieve steady precession without nutation, we need specific precession rate that gives $U^\prime(\theta ) = 0$. In gyroscope limit of fast rotation around the 3-axis, two rates of steady precession are possible:
\[
    \dot{\phi} \approx \frac{mgh}{J_3}, \quad 
    \dot{\phi } \approx \frac{J_3}{I_1 \cos \theta } = \Omega_s, 
\]
where the initial condition $\dot{\phi_0}$ determines in which precession rate the gyroscope ends up. However, gyroscope questions are most realistically tackled by using the gyroscope equation (which is a straightforward application of Eq.\eqref{eq:rate-of-change})
\[
    \boxed{\mathbf{G} = \boldsymbol{\mathbf{\Omega}} \times \mathbf{L},}
\]
where $\mathbf{\Omega}$ is the precession angular velocity. 

\section{Lagrangian and Hamiltonian Mechanics}
\subsection{Euler-Lagrange Equation}
It is usually easier to derive the equation of motion of a system on each particle's position $\{\mathbf{r_k}\}$ by the energy method so that we do not need to add vectors. The normal consideration on conservation of energy relies on separability in the $\dot{x}$ and $\ddot{x} $ terms. Thus, we need to work in mutually independent generalised coordinates $\{q_i\}$ with holonomic transform $\mathbf{r_k} = \sum_{i=1}^{n} q(q_1, q_2, \ldots, q_n ).$

From Newton's second law, for a set of virtual displacement $\{\delta \mathbf{r_k}\}$ compatible with the constraints,
\begin{equation}
    \sum\limits_{k=1}^{N} (m_k \ddot{\mathbf{r_k}} - \mathbf{F_k}) \cdot \delta \mathbf{r_k} = 0.  
    \label{eq:dlambert-thm}
\end{equation}
Our goal is to take the summation off, like what is done in deriving the Euler-Lagrange equation in variational calculus, which requires orthogonality of the coordinates. It can be argued that a set of independent generalised coordinates $\{q_i\}$ can always be obtained such that 
\begin{align*}
    \sum\limits_{k=1}^{N} m_k \ddot{\mathbf{r_k}}  \cdot \textcolor{red}{\delta \mathbf{r_k}} 
    &= \sum_{i,k} m_k \ddot{\mathbf{r} } \cdot  \textcolor{red}{\frac{\partial \mathbf{r} }{\partial q_i} \delta q_i }
    = \sum_{i,k} \left[ \frac{\mathrm{d}}{\mathrm{d}t} \left( m_k \dot{\mathbf{r_k}}\cdot \frac{\partial \mathbf{r_k} }{\partial q_i}  \right) - m_k \dot{\mathbf{r_k} } \cdot \frac{\mathrm{d}}{\mathrm{d}t} \left(\frac{\partial \mathbf{r_k}}{\partial q_i}  \right)\right],
\end{align*}
where the second line follows from the product rule of differentiation. Since the coordinate transformation $\mathbf{r_k} = \mathbf{r_k}(q_1, q_2, \ldots, q_n ) $ is time independent, the chain rule gives
\[
    \dot{\mathbf{r_k}} = \frac{\mathrm{d}\mathbf{r_k} }{\mathrm{d}t} = 
    \sum_{i=1}^{n} \frac{\partial \mathbf{r_k}}{\partial q_i} \dot{q_i} 
    \implies  
    \frac{\partial \dot{\mathbf{r_k} }}{\partial \dot{q_i} } =  \frac{\partial \mathbf{r_k} }{\partial q_i};
    \qquad 
    \frac{\mathrm{d}}{\mathrm{d}t} \left( \frac{\partial \mathbf{r_k} }{\partial q_i} \right)
    = \sum_{j=1}^{n} \frac{\partial ^{2} \mathbf{r_k} }{\partial q_i \partial q_j} \dot{q_j} 
    = \frac{\partial \dot{\mathbf{r_k} } }{\partial q_i},
\]
which essentially states that $\mathrm{d} / \mathrm{d} t $ and $\partial / \partial q_i$ commutes due to the holonomicity. Noting that the total kinetic energy is 
\[
    T = \frac{1}{2} \sum_{k=1}^{N} m \dot{\mathbf{r_k} } \cdot \dot{\mathbf{r_k} } 
    = \frac{1}{2} \sum_{i,j,k}m_k \frac{\partial \mathbf{r_k}}{\partial q_i} \cdot \frac{\partial \mathbf{r_k}}{\partial q_j} \dot{q_i} \dot{q_j},
\]
\[ \implies 
    \frac{\partial T}{\partial \dot{q_i} } = 
    2 \cdot \frac{1}{2} \sum_{j,k} m_k \frac{\partial \mathbf{r_k} }{\partial q_i} \cdot \textcolor{red}{\frac{\partial \mathbf{r_k}}{\partial q_j} \dot{q_j}}
    = \sum_{k=1}^{N} m_k \textcolor{red}{\dot{\mathbf{r_k}}} \cdot \frac{\partial \mathbf{r_k} }{\partial q_i };
    \qquad 
    \frac{\partial T}{\partial q_i} = \sum_{k=1}^{N} m_k \dot{\mathbf{r_k}} \cdot \frac{\partial \dot{\mathbf{r_k}}}{\partial q_i},  
\]
where the factor of $2$ comes from double counting in the symmetric expression of $T.$ The first term in Eq.\eqref{eq:dlambert-thm} can thus be converted to
\begin{align*}
    \sum_{k=1}^{N} m_k \ddot{\mathbf{r_k} } \cdot \delta \mathbf{r_k} 
    &= \sum_{i,k} \left[ \frac{\mathrm{d}}{\mathrm{d}t} \left( \textcolor{blue}{m_k \dot{\mathbf{r_k}}\cdot \frac{\partial \mathbf{r_k} }{\partial q_i}}  \right) - \textcolor{red}{m_k \dot{\mathbf{r_k} } \cdot \frac{\mathrm{d}}{\mathrm{d}t} \left(\frac{\partial \mathbf{r_k}}{\partial q_i} \right)}\right] \\
    &= \sum_{i,k} \left[ \frac{\mathrm{d}}{\mathrm{d}t} \left( \textcolor{blue}{m_k \dot{\mathbf{r_k}}\cdot \frac{\partial \mathbf{r_k} }{\partial q_i}}  \right) - \textcolor{red}{m_k \dot{\mathbf{r_k} } \cdot \frac{\partial \dot{\mathbf{r_k} } }{\partial q_i} }\right] \\
    &= \sum_{i=1}^{n} \textcolor{blue}{\frac{\mathrm{d}}{\mathrm{d}t} \left( \frac{\partial T}{\partial \dot{q_i} } \right)} - \textcolor{red}{\frac{\partial T}{\partial q_i} }.
\end{align*}
If we define \textbf{generalised forces} $Q_i$ such that
\[
    \sum_{k=1}^{N} \mathbf{F_k} \cdot \delta \mathbf{r_k}
    = \sum_{i=1}^{n} Q_i \delta q_i 
    \implies  
    \boxed{Q = \sum_{k=1}^{N} \mathbf{F_k} \cdot \frac{\partial \mathbf{r_k} }{\partial q_i},}  
\] 
then Eq.\ref{eq:dlambert-thm} turns into
\[
    \sum_{i=1}^{n} \left[ \frac{\mathrm{d}}{\mathrm{d}t} \left( \frac{\partial T}{\partial \dot{q_i} } \right) - \frac{\partial T}{\partial q_i} -Q_i = 0\right]
    \implies 
    \boxed{\frac{\mathrm{d}}{\mathrm{d}t} \left( \frac{\partial T}{\partial \dot{q_i} } \right) - \frac{\partial T}{\partial q_i} -Q_i = 0.}
\]
due to independence of $\{q_i\}.$ If all the applied forces are conservative such that 
\[
    V = \sum_{k=1}^{N} \phi_k,\quad \mathbf{F_k} = -\nabla_k \phi_k \implies 
    Q_i = -\frac{\partial V}{\partial q_i}, 
\]
we obtain the Euler-Lagrange equation for conservative $V= V(q_1, q_2, \ldots, q_n )$:
\[
    \boxed{ 
        \frac{\mathrm{d}}{\mathrm{d}t} \left( \frac{\partial \mathcal{\MakeUppercase{L}} }{\partial \dot{q_i} } \right) - \frac{\partial \mathcal{\MakeUppercase{L}} }{\partial q_i} =0, \quad i = 0, 1, \ldots , n, 
    }
\]
where $\mathcal{\MakeUppercase{L}} = T - V$ is the Lagrangian. From calculus of variation, this is equivalent to extremise the action functional $\int_{t_1}^{t_2} \mathcal{\MakeUppercase{L}} \mathrm{d} t$ (Hamilton's principle). Holonomic constraints of the coordinates can be introduced by the Lagrange multipliers. Further conservation could be obtained from Noether's theorem. 

(The derivations are non-examinable)

If any of the $Q_i$ is zero, we get a conserved quantity 
\[
    \frac{\partial \mathcal{\MakeUppercase{L}} }{\partial q_i} = \mathrm{const}.  
\]
If the Lagrangian is independent of time, then we obtain a first integral 
\[
    E = \sum_{i=1}^{n} \dot{q_i} \frac{\partial \mathcal{\MakeUppercase{L}} }{\partial \dot{q_i} } - \mathcal{\MakeUppercase{L}}  = \mathrm{const}, 
\]
which remains under a change of generalised coordinates $\{q_i\} \to \{Q_i(q_1, q_2, \ldots , q_n )\}.$
\subsection{Hamiltonian Mechanics}
In Lagrangian mechanics, the Lagrangian is regarded as a function of independent $\{q_i\}$ and $\{\dot{q_i}\}.$ A Legendre transform involving the conjugate momentum 
\[
    p_i = \frac{\partial \mathcal{\MakeUppercase{L}} }{\partial q_i} 
\]
gives Hamiltonian 
\[
    \boxed{ 
        H(q_i, p_i, t) = 
        \sum_{i=1}^{n} p_i \dot{q_i}(p_i) - \mathcal{\MakeUppercase{L}}(q_i, \dot{q_i}(p_i),t ). 
    }
\]
From this transform and the chain rule, we have the Hamilton's equations of motion
\[
    \dot{q_i} = \frac{\partial H}{\partial p_i}, \quad \dot{p_i} = -\frac{\partial H}{\partial q_i}, \quad 
    -\left( \frac{\partial L}{\partial t}   \right)_{q, \dot{q} } = \left(\frac{\partial H}{\partial t}\right)_{q,p}. 
\]
Thus, if $\mathcal{\MakeUppercase{L}} $ is independent of time, $H$ is conserved and equals the total energy. Note that $H$ is an explicit function of $p_i,$ which does not explicitly depend on $q_i.$ This symmetric ground is not provided in the Lagrangian formalism. 

\section{Normal Modes}
Most many-body problems do not have analytical solutions, with the notable exception of quadratic interaction. In such a system, the total energy could be expressed as a sum of quadratic forms:
\[
    T = \dot{\mathbf{x} } \cdot \overline{\overline{M}} \cdot \dot{\mathbf{x} }, \quad 
    U = \mathbf{x} \cdot \overline{\overline{K}} \cdot \mathbf{x}, 
\]
where the tensors $\overline{\overline{M}} $ and $\overline{\overline{K}} $ are constructed to be symmetric. 
\end{document}
