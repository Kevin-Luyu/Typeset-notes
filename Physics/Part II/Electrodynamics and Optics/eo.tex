\documentclass{article}
\usepackage[thmmarks, framed]{ntheorem} % ntueorem must come before amsmath, or the cross-reference will not be working
\usepackage[utf8]{inputenc}
\usepackage{graphicx}
\usepackage{booktabs}
\usepackage[a4paper, portrait, margin=1in]{geometry}
\usepackage{amsmath}
\usepackage{amsfonts}
\usepackage{mathtools}
\usepackage{physics}
\usepackage{xcolor}
% needed for framed theorems
\usepackage{framed} % or, "mdframed"
%
\usepackage{bm}
\newcommand{\tn}[1]{\underline{\underline{\mathbf{#1}}}} % a tensor
\newtheorem{theorem}{Theorem} 
\newframedtheorem{frm-thm}{Theorem} %needed for framed theorems 
\theorembodyfont{\upshape}
\newframedtheorem{frm-res}{Result}
\theorembodyfont{\upshape}
\newframedtheorem{frm-def}{Definition}
\theoremstyle{nonumberplain} 
\theoremheaderfont{\itshape}
\theorembodyfont{\normalfont}
\theoremsymbol{\ensuremath{\square}}
\newtheorem{proof}{Proof}
\title{Electrodynamics and Optics}
\date{Michaelmas 2022}
\author{Yu Lu}
\begin{document}
\maketitle
\section{Preliminary knowledge}
Classical electrodynamics is governed by the four Maxwell's equations
\begin{equation}    
    \begin{aligned}
        \nabla \cdot \mathbf{D} &= \rho  \\
        \nabla \cdot \mathbf{B} &= 0 \\
        \nabla \times \mathbf{E} &= -\dot{\mathbf{B} } \\ 
        \nabla \times \mathbf{H} &= \mathbf{J} + \dot{\mathbf{D} }, 
    \end{aligned}
\end{equation}
where $\mathbf{D}$ and $\mathbf{B} $ are the electric/magnetic flux density while $\mathbf{E} $ and $\mathbf{H} $ are the corresponding fields. These are related via 
\(
    \mathbf{D} = \epsilon_0 \tn{\epsilon } \cdot \mathbf{E}, 
\quad  \mathbf{B} = \mu_0 \tn{\mu} \cdot \mathbf{H}, 
\) 
where $\tn{\epsilon } $ and $\tn{\mu} $ are Hermitian tensors (thus diagonalisable) of the second rank. 

In the absence of charge and current, the EM field can propagate as harmonic waves
\[  
    \mathbf{E} = \mathbf{E_0} e^{i (\mathbf{k} \cdot \mathbf{r} - \omega  t)}, \quad 
    \mathbf{B} = \mathbf{B_0} e^{i (\mathbf{k} \cdot \mathbf{r} - \omega  t)},
    \] 
satisfying 
\(
    \mathbf{k} \cdot D = 0, \quad \mathbf{k} \cdot \mathbf{B} =0, \quad \mathbf{k} \times \mathbf{E} = \omega \mathbf{B} , \quad \mathbf{k} \times \mathbf{H} = - \omega \mathbf{D}. 
\) 
The wave also carries an energy density $u = \frac{1}{2} \mathbf{E} \cdot \mathbf{D} + \frac{1}{2} \mathbf{B} \cdot \mathbf{H} $ and the energy flux is $\mathbf{N} = \mathbf{E} \times  \mathbf{H}.$
\section{Optics}
The Maxwell's equations are linear, making it possible to investigate polarised orthogonal components of fields separately. 
\subsection{Jones Notation}
\end{document}