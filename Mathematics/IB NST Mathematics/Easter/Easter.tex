\documentclass{article}
\usepackage[thmmarks, framed]{ntheorem} % ntheorem must come before amsmath, or the cross-reference will not be working
\usepackage[utf8]{inputenc}
\usepackage{graphicx} 
\usepackage{booktabs}
\usepackage[a4paper, portrait, margin=1in]{geometry}
\usepackage{amsmath}
\usepackage{amsfonts}
\usepackage{mathtools}
\usepackage{physics}
\usepackage{xcolor}
\usepackage{caption}
\usepackage{subcaption}
% needed for framed theorems
\usepackage{framed} % or, "mdframed"
\usepackage{bm}
\newframedtheorem{frm-res}{Result}
\theorembodyfont{\upshape}
\title{NST IB Mathematical Methods III}
\date{Easter, 2023}
\author{Yu Lu}
\begin{document}
\maketitle
\tableofcontents
\section{Normal modes}
An $n$-particle system coupled by spring-like forces with a Lagrangian
\[
    \mathcal{\MakeUppercase{l}} = \frac{1}{2} T_{ij} \dot{\theta_i} \dot{\theta_j} - \frac{1}{2} V_{ij} \theta_i \theta_j = \frac{1}{2} \mathbf{\dot{q} ^{\top}} \mathbf{T} \mathbf{\dot{q} } -   \frac{1}{2} \mathbf{q ^{\top}} \mathbf{V} \mathbf{q }, 
\]
where $T_{ij}, V_{ij} $ are constants, has an equation of motion
\begin{equation}
    \label{eq:normal-modes}
    \mathbf{T} \ddot{\mathbf{q} } = - \mathbf{V} \mathbf{q}. 
\end{equation}

Eq.\eqref{eq:normal-modes} admits $n$ linearly independent normal modes $q(t)$ such that
\[
    q_i(t) = Q_i \sin( \omega_i t - \phi_i), 
\]
where $Q_i$ and $\phi_i$ are undetermined constants while $\omega_i$ are the normal modes frequencies. From the form of the guessed solution, the angular frequency of modes $\omega_i$ can be found through the generalised eigenvalue problem 
\[
    \boxed{
        (-\omega ^{2} \mathbf{T} + \mathbf{V} ) \mathbf{Q} = \textbf{0},
    }
\]
where $Q^{(1)}, \ldots Q^{(N)} $ are generalised eigenvectors satisfying the \textbf{orthogonal} relation
\[
    (\mathbf{Q}^{(i)})^{\top} \mathbf{T} Q^{(j)} = 0, \quad i \neq j. 
\]

The general solution of the system is
\[
    \mathbf{q}(t) = \sum_{m=1}^{N} A^{(m)} \mathbf{Q}^{(m)} \sin(\omega_m t - \phi_m),
\]
where $A^{(m)}$ are undertermined coefficients from the initial conditions. From the orthogonality conditions, it can be shown that a linear combination of generalised coordinates $q_1, q_2, \ldots $ oscillates at a single frequency $\omega_i$ of the i-th eigenvector $Q^{(i)}$
\[
    \boxed{\alpha^{(n)}(t) = q_i (t) T_{ij} Q_{j}^{(n)} } = A^{(n)} \sin(\omega_i t - \phi_i). 
\]
The proof is easy after expanding $\mathbf{q}$ in the eigenvector basis $q_i = \sum_{m} \alpha^{(m)}(t) Q_i^{(m)} $
\section{Groups and representations}
\end{document}